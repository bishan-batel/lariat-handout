
\begin{section}{What is Lariat}

 \begin{displayquote}
	 It is important to understand that this is not the only way to solve this
	 assignment, merely the way that worked best for me.
 \end{displayquote}

 \indent Under the hood, Lariat looks like a "linked list of arrays".

 \indent The main goal is to have a more efficient (compared to array) insert in the
 middle.
 Reminder: if array (and consequently \mintinline{cpp}{std::vector}) insert an element
 in the middle, all elements to the right of the insert position have to be
 shifted 1 position to the right to free a spot for new element. This is $O(N)$ (linear) operation. Another possible problem is if there is not \uwave{anough}\footnote{enough} space,
 which will required reallocation if the whole array.
 Lariat solves the problem by storing small segments of contiguous memory non-contiguously.


 During insert one of $2$ things can happen (see \uwave{datail}\footnote{detail} later):
 \begin{itemize}
	 \item the contiguous block may have extra space, so only some elements have to be shifted
	       \begin{minted}[autogobble]{cpp}
//    0  1            2  3  4        // index
// +------------+  +------------+
// |  1  3     -|->|  4  5  6  -|->  // data
// +------------+  +------------+
// after insert 2 before 3
//    0  1  2         3  4  5         6  7  8
// +------------+  +------------+  +------------+
// |  1  2  3  -|->|  4  5  6  -|->|  9  9  9  -|->
// +------------+  +------------+  +------------+
// only value 3 had to shifted
	\end{minted}

	 \item block containing insert position is full, then we split it
	       \begin{minted}[autogobble]{cpp}
// insert 8 before 5
// first split:
//    0  1  2         3  4            5  6            6  7  8       
// +------------+  +------------+  +------------+  +------------+  
// |  1  2  3  -|->|  4  8     -|->|  5  6     -|->|  9  9  9  -|->
// +------------+  +------------+  +------------+  +------------+  
// now insert
//    0  1  2         3  4            5  6            7  8  9       
// +------------+  +------------+  +------------+  +------------+    
// |  1  2  3  -|->|  4  8     -|->|  5  6     -|->|  9  9  9  -|->
// +------------+  +------------+  +------------+  +------------+    
// only values 5 and 6 were copied
			\end{minted}
 \end{itemize}


\end{section}
