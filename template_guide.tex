\begin{section}{Luke's Guide to Templated Templates}

 \begin{displayquote}
	 "We put some templates in your templates so you could generate code at compile
	 time while you generate code at compile time."
	 -C++
 \end{displayquote}

 \indent A template is a section of code like any other that gets "written" at compile
 time, rather than when you actually write templated code. In this sense,
 writing a template is about using the compiler to your own advantage to avoid
 doing unnecessary work.

 \begin{displayquote}
	 Laziness FTW
 \end{displayquote}


 A normal function template is declard as such:
 \begin{minted}[autogobble]{cpp}
		template<typename T>
		void Foo(T bar);
	\end{minted}

 The definition requires knowledge of the template parameters in order to work:

 \begin{minted}[autogobble]{cpp}
		template<typename T>
		void Foo<T>(T bar) {
			//     |
			//      `-[function template ]

			//...
		}
	\end{minted}

 \indent In the Lariat assignment, you will have to make a templated copy constructor
 and assignment operator for the Lariat template class. This is similar to
 writing a static function template, but it's written within another
 encompassing template, the Lariat class in this case.

 \indent The basic declaration of the function is just like a normal template
 function, but within a class scope.

 \begin{minted}[autogobble]{cpp}
		template<typename T>
		class Foo {
			//...

			template<typename Bar>
			void Baz(Bar & qux);

			//...
		};
	\end{minted}

 \indent The definition of the function takes a little bit more work, which may be confusing at first.


 \begin{minted}[autogobble]{cpp}
    template<typename T>    // Gives the external class template information
    template<typename Bar>  // Gives the internal function template information
		void Foo<T>::Baz<Bar>(Bar & qux) {
      // |          |
      // |           `-[function signature]
      //  `-[class namespace]

      // ...
    }
 \end{minted}


 \indent The first template declaration belongs to the class, which may be instantiated
 multiple times with multiple name-mangled signatures based on the compiler's
 implementation of templates.

 \indent The second template declaration belongs to the function, which will likely be
 instantiated multiple times for every single class instantiation. In this way
 the structure resembles a tree:


 \begin{figure}[h]
	 \centering
	 \begin{verbatim}
										+-source------------------------------+
										| original code written by programmer |
										+-------------------------------------+
											 /                               \
					 +-class---------+                     +-class---------+
					 | instantiation |                     | instantiation |
					 | for type A    |                     | for type B    |
					 +---------------+                     +---------------+
					 /               \                     /               \
+-function------+ +-function------+      +-function------+ +-function------+
| type A        | | type B        |      | type A        | | type B        |
| instantiation | | instantiation |      | instantiation | | instantiation |
+---------------+ +---------------+      +---------------+ +---------------+
	\end{verbatim}
	 \caption{\#dat\_ascii}
 \end{figure}



 \pagebreak
 \indent The next little template trick you might run into with this assignment comes
 when you try to refer to internal member types in the instantiation of code.
 I personally used this in my helper functions that returned pointers to nodes
 in the list, but it's useful for any further template work you might be
 interested in doing.

 \indent Dependent types are also instantiated for ech class, with the exact same
 process as function templates. This means that using just the internal name
 in external code will lead to type confusion, where the compiler isn't sure
 which owning class is being referred to.

 \indent A class class template with a member type and a member function returning a
 pointer to the member type:

 \begin{minted}[autogobble]{cpp}
    template<typename T>
    class Foo {
      //...

      struct Bar {
        //...
      }

      Bar * Baz();

      //...
    };
 \end{minted}

 The member function here must be defined as such:
 \begin{minted}[autogobble]{cpp}

    template<typename T>
		typename Foo<T>::Bar * Foo<T>::Baz() {
      // |       |    |       |     |
      // |       |    |       |      `-[function signature]
      // |       |    |        `-[class namespace]
      // |       |     `-[member typename]
      // |        `-[class namespace]
      //  `-[declaration of token as user-defined type]

      //...
    }
 \end{minted}

 \pagebreak
 \indent The last real trick I want to share with you pertains directly to the copy
 constructors this assignment requires. At their instantiation, class templates
 only have direct access to their own instatiation's member variable. This
 means that in order to see another instantiation's private variables, there
 must be a declaration of each instantiation as a friend.

 \begin{displayquote}
	 How do you make something to account for any scenario at compile time?

	 Simple. More templates.

	 Declare a friend class template in your original class template.

	 \begin{minted}[autogobble]{cpp}
    template<typename T>>
    class Foo {
      //...

      template<typename U>  //[nested template declaration]
      friend class Foo<U>;  //[friend declaration for class template]

      //...
    }
	 \end{minted}
 \end{displayquote}

 \indent This has been my brief introduction to templates as they pertain to this
 assignment. they may seem scary, but templates allow for a greater depth and
 elegance of code, decreasing the total writing time and reducing the volume of
 duplicated code.

\end{section}
