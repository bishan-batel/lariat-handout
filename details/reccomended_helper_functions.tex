\subsection{Recommended Helper Functions}
\label{section:rec_helper}


\subsubsection{\cpp{split}}

This helper function takes a full node and splits into two nodes of an
aproximately equivalent number of elements.

\begin{enumerate}
	\item First, create a new node to split the original node into.

	\item Calculate the index from which to move elements over from. This should
	      account for an element being appended to the second node after the
	      function has been called. If you're implementing this with overflow,
	      you should have already worked out how the overflow gets assigned its
	      position.

	\item Loop through the elements from the index in the first node, transferring
	      them each to the newly added node. Update the added node's count during
	      the loop.

	      \begin{minted}[autogobble]{cpp}
				/*
					(0) Initial state, with the full node that was passed in.

					+-+-node----------+---+      +-+-
					|#|[a][b][c][d][e]|(5)| ---> |#|[
					+-+-0--1--2--3--4-+---+      +-+-

					(1) The second node has been created and elements are being transferred
					to it.

					[index = 3]
					|
					+-+-node----------+---+      +-+-
					|#|[a][b][c][d][e]|(5)| ---> |#|[
					+-+-0--1--2--3--4-+---+      +-+-
					________|
					|
					+-+-added---------+---+
					|#|[_][_][_][_][_]|(0)| ---> [NULL]
					+-+-0--1--2--3--4-+---+

					(2) The elements have been transferred to the added node and the count of
					the added node has been updated.

					+-+-node----------+---+      +-+-
					|#|[a][b][c][_][_]|(5)| ---> |#|[
					+-+-0--1--2--3--4-+---+      +-+-

					+-+-added---------+---+
					|#|[d][e][_][_][_]|(2)| ---> [NULL]
					+-+-0--1--2--3--4-+---+
				*/
		\end{minted}

	\item Set the count of the original node to index, and insert the added node
	      immediately after the original node.
\end{enumerate}



\subsubsection{\cpp{findElement}}

\indent This function takes a global index to find in the deque and must return
both a pointer to the node in the list and the local index of the element in the returned node.
Clearly returning multiple variables is a  \textit{small} challenge in C and C++,
so you will have to find a way to work around that.
The algorithm for this function is focused around looping through the list
and tracking the total number of elements stored before the current
node. I used a counter and added each node's count until I passed the
desired global index.

\indent Because each node in the list only stores the number of elements it is
using actively, the count of each has to be added individually to keep an
accurate count of the number of elements that have been stepped over.


\subsubsection{shiftUp}
\indent This helper function is almost identical to one that was particularly
useful for the CS170 Vector lab. It uses a simple swap algorithm to swap
every element between the index and the last element in the node's array
in place.

\indent Make sure your bound-checking is good and you have accurately defined the
limits of your memory. If you find a wrong value somewhere unexpected,
you probably have the index limit wrong somehow.

\pagebreak
\begin{minted}[autogobble]{cpp}
/*
  (0) Initial state of the node passed in. Shifting everything right from
      the third element.

          [index = 2]
               |
       +-node----------------------+
       |[a][b][c][d][e][f][g][h][i]|
       +-0--1--2--3--4--5--6--7--8-+

  (1) First pass of the loop, swapping the index with the loop count.

    [index = 2]   [lc = 3]
              \    /
       +-node----------------------+
       |[a][b][d][c][e][f][g][h][i]|
       +-0--1--2--3--4--5--6--7--8-+

  (2) Last pass of the loop, swapping the index with the last element.
      Notice that the third element, the index, now holds the value that
      would have been pushed off the end of the array. This is useful for
      tracking the last element that would have been lost.

    [index = 2]                  [lc = 3]
              \                   /
       +-node----------------------+
       |[a][b][i][c][d][e][f][g][h]|
       +-0--1--2--3--4--5--6--7--8-+
                \
               [overflow element]
*/
\end{minted}

I would highly recommend writing a templatized swap function. It has uses
in most basic algorithms related to data structures.


\subsubsection{\cpp{shiftDown}}
\indent The shift down helper function works very similarly to the shift up, by
swapping values in the node's array.
The key difference between the two is that the shift down function does
not have a fixed index. It should move down the array, swapping each
element with the element immediately before it. In this way, it still
preserves the data it is writing over, but does so by shifting it from
the start of the range to the end rather than the opposited direction.
