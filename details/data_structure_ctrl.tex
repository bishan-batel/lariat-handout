\subsection{Data Structure Control}

\subsubsection{\cpp{clear}}
\indent This is a relatively simple function with a similar algorithm to the
class destructor.

\begin{itemize}
	\item First, loop through the list, freeing each node in turn.
	\item Once the list is empty, update the necessary member variables.
\end{itemize}

\subsection{\cpp{compact}}
\indent Compact takes all the data stored in the linked list and moves it into the
smallest number of nodes possible. Then it frees all empty nodes at the
end of the list.

The algorithm for this walks the list at two points in parallel. I will
refer to the walker having elements shifted into it as the left foot and
the walker reading through the elements to shift as the right foot.

\begin{itemize}
	\item First, loop through the list with both feet until the left node is not
	      already full and the right foot hasn't walked off the list.
	\item Next, walk through the list while the right foot hasn't lost the list.

	\item This loop should first store the count of elements in the right node,
	      then set the count of the right node to zero. This allows the left foot
	      to update it as it is given elements to store.
	\item This loop should have a nested loop that stores each value from the right
	      foot in the left foot's node, and steps the left foot to the next node
	      when it is filled.
\end{itemize}


\pagebreak

\begin{verbatim}
 ______________________________________  
|.-KEY--------------------------------.| 
|| [node]   [values]   [count]        || 
||   |         |           |  [next]  || 
||   v         v           v    |     || 
||  +-+-node-------------+---+  v     || 
||  |#|[_][_][_][_][_][_]|(_)| --->   || 
||  +-+------------------+---+        || 
||      0  1  2  3  4  5  <-[indices] || 
||____________________________________|| 
'--------------------------------------' 
\end{verbatim}


\begin{enumerate}

	\item The starting position of the inner loop, transfering right's first
	      element into left's fifth space

	      \begin{verbatim}
                                 [rpos = 0]                           
                                     |                      [rtot = 3]
 +-+-left-------------+---+      +-+-right------------+---+      +-+-...
 |0|[a][b][c][d][_][_]|(4)| ---> |1|[e][f][g][_][_][_]|(0)| ---> |2|[...
 +-+------------------+---+      +-+------------------+---+      +-+-...
     0  1  2  3  4  5                0  1  2  3  4  5                 					
			 \end{verbatim}


	\item Next important point, the left foot is now full. Set the left foot
	      to the next node in the list and continue the algorithm
	      \begin{verbatim}
                                       [rpos = 2]                     
                                           |                [rtot = 3]
 +-+-left-------------+---+      +-+-right------------+---+      +-+-...
 |0|[a][b][c][d][e][f]|(6)| ---> |1|[_][_][g][_][_][_]|(0)| ---> |2|[... 
 +-+------------------+---+      +-+------------------+---+      +-+-...				
     0  1  2  3  4  5                0  1  2  3  4  5                 				
			 \end{verbatim}


	\item Left and right feet are now standing on the same node, but the
	      algorithm still works because it is based on index in the node
	      rather than the node itself.
	      \begin{verbatim}
        [rpos = 2]                                                     
            |                                                [rtot = 3]
  +-+-left/right-------+---+      +-+-right->next------+---+      +-+-... 
  |1|[_][_][g][_][_][_]|(0)| ---> |2|[h][i][j][k][_][_]|(4)| ---> |3|[... 					
  +-+------------------+---+      +-+------------------+---+      +-+-...					
      0  1  2  3  4  5                0  1  2  3  4  5                 				
		\end{verbatim}


	      \pagebreak
	\item The left foot has had the element at rpos shifted down, and the
	      right foot has stepped to the next node. Note that rtot has been
	      updated to 4, the right node has had its count zeroed out, and
	      rtot has been updated to the right node's original count.


	      \begin{verbatim}
                                    [rpos = 1]                        				
                                        |                   [rtot = 4]
 +-+-left-------------+---+      +-+-right------------+---+      +-+-... 
 |1|[g][h][_][_][_][_]|(2)| ---> |2|[_][i][j][k][_][_]|(0)| ---> |3|[...
 +-+------------------+---+      +-+------------------+---+      +-+-...
     0  1  2  3  4  5                0  1  2  3  4  5                 
				\end{verbatim}


\end{enumerate}

Once the deque has been compacted, remove all the extra nodes from the
end of the list.
