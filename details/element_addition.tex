\subsection{Element Addition}


\subsubsection{\cpp{insert}}
\label{func:insert}

\indent Insert an element into the data structure at the index, between the element
at \cpp{[index - 1]} and the element at \cpp{[index]}
The first thing to this function is to check for an Out of Bounds error. If
the index is invalid, throw a \cpp{LariatException} with \cpp{E_BAD_INDEX} and the
description "Subscript is out of range"
Make sure to handle the "edge" cases, which allow for insertion at the end
of the deque as well as the beginning. I personally suggest calling
\cpp{push_back} and \cpp{push_front} in these cases, as it helps minimize the amount
of code written and the algorithm is identical anyways.
\indent The next thing to do is to set up the actual insertion algorithm.
First, find the node and local index of the element being inserted. This
can be done with the \cpp{findElement} helper function detailed in the
Recommended Helper Functions section of this guide.

Next, shift all elements past that local index one element to the right.
This can be done with the \cpp{shiftUp} function detailed in \fullref{section:rec_helper}.

\begin{minted}[autogobble]{cpp}
      /*
            [e]
             |
             v
        +------------------+      (0) Starting position, inserting 'e' at
        |[a][b][c][d][_][_]|          index 1 of the node
        +-0--1--2--3--4--5-+

            [e]
             |
             v
        +------------------+      (1) Shift all elements right from index
        |[a][_][b][c][d][_]|
        +-0--1--2--3--4--5-+

            [_]
             |
             v
        +------------------+      (2) Set the newly open element to the value
        |[a][e][b][c][d][_]|          being inserted, 'e' here
        +-0--1--2--3--4--5-+
      */
\end{minted}


\pagebreak

\indent This should work for the most common use-case, but it will cause a
problem if the node is full.
If the node is full, you will need to keep track of the last element in
the node before you shift all the elements to the right. Again, this is
easy to do with the recommended helper function.
I call this "popped off" element the overflow.

\begin{minted}[autogobble]{cpp}
      /*
            [g]
             |            [_]
             v            /
        +------------------+      (0) Starting position, inserting 'g' at
        |[a][b][c][d][e][f]|          index 1 of a full node
        +-0--1--2--3--4--5-+

            [g]
             |            [f] <-[overflow]
             v            /
        +------------------+      (1) Shift all elements right from index,
        |[a][_][b][c][d][e]|          keeping track of the overflow, 'f' here
        +-0--1--2--3--4--5-+

            [_]
             |            [f]
             v            /
        +------------------+      (2) Set the newly open element to the value
        |[a][g][b][c][d][e]|          being inserted, 'g' here
        +-0--1--2--3--4--5-+
      */
\end{minted}

\indent Next, you will need to split the node.
I would recommend writing a helper function for this algorithm as it
will be used elsewhere. I have detailed the split algorithm in \fullref{section:rec_helper}.
You will need to transfer the overflow to the last element of the new
node created by the split. It is possible to put this part of the
algorithm directly in the split function. I will not detail that in
this guide, but rather leave it for you to discover on your own.
Working out that algorithm will probably make your code much cleaner,
so I would definitely recommend figuring it out.
As the split algorithm accurately updates the node counts for the split
nodes, the only thing left to do is increment the node count.




\subsubsection{\cpp{push_back}}
\indent This is an easy algorithm using the \cpp{split}  function (\ref{section:rec_helper}).

\begin{itemize}
	\item If the tail node is full, split the node and update the \cpp{tail_} pointer.
	\item Set the last element in the tail's array to the value.
	\item Increment the tail node's count.
\end{itemize}

\pagebreak
\subsubsection{\cpp{push_front}}
This algorithm is more similar to the \cpp{insert} function (\ref{func:insert}) than the \cpp{push_back}
algorithm, but is still relatively simple.

\begin{itemize}
	\item If the head node is empty, increment the node's count.
	\item If the head node is full, you will need to shift the elements up, in the
	      same way they were shifted in the insert function, making sure to track the overflow.
	\item Next you will need to split the node.
	      In order to account for splitting the only node in the linked list, you
	      will have to update the \cpp{tail_} pointer as necessary.
	\item If the head node isn't full yet, just shift the head node up an element
	      from element 0 and increase the count.
	\item Set the 0'th element of the head to the value.
\end{itemize}
