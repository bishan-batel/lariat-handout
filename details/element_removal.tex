\subsection{Element Removal}


\subsubsection{\cpp{erase}}

This function uses the \cpp{findElement} helper function I have detailed in
\fullref{section:rec_helper}. Having implemented that, the function itself is relatively simple.
You can use the \cpp{pop_back} and \cpp{pop_front} functions if the index requested is
the first or last element.


\begin{itemize}
	\item First, find the containing node and local index of the requested global
	      index.
	\item Shift all the elements in the node beyond the local index left one element,
	      covering the element being erased. This is done with the \cpp{shiftDown}
	      function detailed in \fullref{section:rec_helper}. Make sure
	      to account for the node being only one element large.
	\item Decrement the node's count.
\end{itemize}


\subsubsection{\cpp{pop_back}}
Decrement the count of the tail node.

\subsubsection{\cpp{pop_front}}
Shift all elements in the head node down one element.
Decrement the head's count.

If the head node is now empty, free the associated memory.
