\subsection{Element Access}

\indent As the const and non-const versions of these functions are algorithmically
and syntactically identical, I would recommend (may \spellerr{Shilling}{Schilling \includegraphics[width=0.02\textwidth]{schilldawg.jpg}} forgive me)
copy\/pasting the code from one to the other. You can't call one from inside
the other, as that would violate the const requirements.

\subsubsection{\cpp{operator[]}}
Find the containing node and local index of the index passed in. Like
insert and erase, this is easily done with the \cpp{findElement} helper
function I detailed in \fullref{section:rec_helper}.

Return the element at the local index of the containing node.

\subsubsection{\cpp{first}}
This is one of the easiest functions in this assignment.

Return the first element of the head node.

\subsubsection{\cpp{last}}
This is also an easy function.

Return the last element in the tail node.
